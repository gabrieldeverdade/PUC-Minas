%%%%%%%%%%%%%%%%%%%%%%%%%%%%%%%%%%%%%
%% Resumo
%% Autor: Fábio Leandro Rodrigues Cordeiro
%% Version: 1.0
%%%%%%%%%%%%%%%%%%%%%%%%%%%%%%%%%%%%%

\begin{newpage}
	\thispagestyle{empty}
	\setlength{\baselineskip}{1.5\baselineskip} % Espacamento: 1.5
	\begin{center}
		\textbf{RESUMO} \\ [1.5\baselineskip]
	\end{center}
	\singlespace
	\noindent 
Este trabalho foi realizado na linguagem Java e apresenta a implementação de três métodos distintos requisitados pelo professor Zenilton Kleber Gonçalves para a identificação de blocos e componentes biconexos em grafos. Um grafo biconexo é caracterizado pela presença de dois caminhos internamente disjuntos entre cada par de vértices, tornando-o mais robusto e tolerante a falhas. Os métodos implementados foram: (i) verificação de dois caminhos internamente disjuntos entre pares de vértices (ou de um ciclo), (ii - DFS), identificação de articulações ao testar a conectividade após a remoção de vértices, e (iii - TARJAN) o algoritmo de Tarjan para a identificação de componentes fortemente conectados.

Foram realizados experimentos utilizando grafos aleatórios com 100, 1.000, 10.000 e 100.000 vértices, nos quais o tempo médio de execução de cada método foi analisado. A verificação de caminhos disjuntos mostrou-se eficiente para grafos menores, enquanto o método de identificação de articulações apresentou maior complexidade para grandes grafos. O algoritmo de Tarjan se destacou por oferecer um equilíbrio entre desempenho e robustez, sendo mais adequado para grafos densos e de grande escala.

Os resultados fornecem uma base sólida para a escolha do método mais apropriado para diferentes cenários de análise de grafos, considerando o tamanho do grafo e a necessidade de robustez e tolerância a falhas.
        \vspace{1.5\baselineskip} 
	\par
        \noindent Palavras-chave: {Grafo Biconexo. Verificação de Caminhos. Componentes Fortemente Conectados. Algoritmo de Tarjan.}% Palavras-chave
\end{newpage}


